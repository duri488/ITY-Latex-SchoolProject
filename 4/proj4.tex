\documentclass[a4paper, 11pt, hidelinks]{article}
\usepackage[slovak]{babel}
\usepackage[left=2cm, top=3cm, text={17cm, 24cm}]{geometry}
\usepackage[utf8]{inputenc}
\usepackage[unicode]{hyperref}
\usepackage{times}
\usepackage{url}


\begin{document}

\urlstyle{same}
\renewcommand{\refname}{Literatúra}

\begin{titlepage}

\begin{center}

\Huge\textsc{Vysoké učení technické v Brně} \\
\huge\textsc{Fakulta informačních technologií} \\	

\vspace{\stretch{0.382}}

\LARGE Typografie a publikování\,--\,4. projekt \\
\Huge Bibliografia

\vspace{\stretch{0.618}}
\end{center}

{\LARGE \today \hfill
Juraj Lazorík (xlazor02)}

\end{titlepage}

\section{Rozpoznávanie reči}
Rozpoznávanie reči je automatický prevod ľudskej reči do textu. Existujú rôzne metódy na analýzu a prevod hlasu. Všetky tieto metódy sú výpočtovo veľmi náročné. Využívajú sa tu napríklad neurónové siete \cite{rekurentNeuronSiet}. Z toho dôvodu sa väčšinou pri tvorbe aplikácií využívajúcich túto službu používajú API rôznych IT gigantov. Ako prebieha spracovanie reči je popísané v \cite{LouizouPhiliposC2007}.

\subsection{História}
História rozpoznávania reči začala v roku 1952, kedy sa trojici vedcov podarilo rozpoznať jednotlivé číslice jedným hovorcom \cite{wikiRozpozReci}. Od tohto začiatku sme sa dostali až do doby, kedy môžeme využívať túto službu každý deň. Najbežnejším využitím dnes sú hlasoví asistenti, ako napríklad Cortana, o ktorej sa viete dozvedieť viac v~\cite{wikiCortana}.

\subsection{Rozpoznávanie reči a jazyky}
Pri rozpoznávaní reči je mnoho problémov. Jedným z nich sú aj jazyky a rôzne dialekty. Podrobnejšie informácie, ako tieto problémy riešime, je popísané v \cite{rozpozRecVybraneJazyky}.

\subsection{Rozpoznávanie reči okolo nás}
Rozpoznávanie reči už nájdeme všade v našom okolí \cite{techHlasKom}. Či už je to mobil alebo počítač. No v dnešnej dobe nezaostávajú ani ostatné spotrebiče, o čom sa môžete viac dočítať v \cite{voiceRecInHome}.

\subsection{Projekty s rozpoznávaním reči}
Keby sme nemali dostupných toľko dát, nič ako hlasoví asistenti by neexistovalo ešte 10 až 15 rokov \cite{talkingToMachines}. Rozpoznávanie sa dá využívať nie len na zjednodušenie práce, ale aj napríklad na rozpoznávanie druhov vtákov v prírode, o čom sa píše v \cite{birdRecog}.

V prípade, že by ste chceli využiť rozpoznávanie vo vašom projekte, môžete využiť jednu z dostupných API. Samozrejme, môžete si naimplementovať vlastný softvér podľa vašich predstáv. Tomuto sa venuje \cite{howToDoSpeech}.

\newpage
\bibliographystyle{czechiso}
\bibliography{proj4}

\end{document}
